\paragraph{}
Au terme de ce projet, notre implémentation permettait à un Nao de
reconnaître différents stimuli visuels ou tactiles et de constuire un
modèle de prédiction réactif et évolutif afin d'anticiper les événements qui
se produisent dans son environnement. Les prédictions et les réactions
débouchaient ensuite sur la production de mouvement et de son, donnant
l'impression que le robot ressentait une émotion.

\paragraph{}
Nos principales difficultés ne se sont pas situé au niveau du modèle
d'apprentissage. Effectivement, celui-ci a demandé un certain travail afin
de trouver quelque chose qui convenait aux effets que nous souhaitions
produire, mais nous avons pu y parvenir. En revanche, la perception de
stimuli et la production d'émotions ont été plus problématiques. 

\paragraph{}
Bien que nous ayons réussi à créer un filtre remplissant les conditions
nécessaires à une bonne exécution du modèle prédictif, nous n'avons pas
réussi à trouver de solutions permettant de capter les stimuli approprié
lors d'une interaction non codifiée avec le Nao.

\paragraph{}
La production d'émotions s'est aussi révélée quelque peu décevante,
effectivement, même si nous avions des idées relativement simples qui ne
nécessitait que l'utilisation des bras, nous n'avons pas réussi à les porter
sur le Nao, à cause des limites des degrés de liberté.
