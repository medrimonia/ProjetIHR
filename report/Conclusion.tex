\paragraph{}
Au terme de ce projet, notre implémentation permettait à un Nao de
reconnaître différents stimuli visuels ou tactiles et de construire un
modèle de prédiction réactif et évolutif afin d'anticiper les événements qui
se produisent dans son environnement. Les prédictions et les réactions
débouchaient ensuite sur la production de mouvement et de son, donnant
l'impression que le robot ressentait une émotion.

\paragraph{}
Nos principales difficultés ne se sont pas situé au niveau du modèle
d'apprentissage. Effectivement, celui-ci a demandé un certain travail afin
de trouver quelque chose qui convenait aux effets que nous souhaitions
produire, mais nous avons pu y parvenir. En revanche, la perception de
stimuli et la production d'émotions ont été plus problématiques. 

\paragraph{}
Bien que nous ayons réussi à créer un filtre remplissant les conditions
nécessaires à une bonne exécution du modèle prédictif, nous n'avons pas
réussi à trouver de solutions permettant de capter les stimuli approprié
lors d'une interaction non codifiée avec le Nao.

\paragraph{}
La production d'émotions s'est aussi révélée quelque peu décevante,
effectivement, même si nous avions des idées relativement simples qui ne
nécessitait que l'utilisation des bras, nous n'avons pas réussi à les porter
sur le Nao, à cause des différentes limites évoquées précédemment.

% Concept appris
\paragraph{}
Ce projet nous a permis de prendre en main des modèles d'apprentissage et
de comprendre à quel point il était complexe de donner un air réaliste à une
interaction homme-robot. Nous avons en particulier pu prendre conscience de
l'importance qu'il était nécessaire d'accorder à différents aspects
mécaniques tels que l'amplitude des degrés de liberté ou encore la
continuité des données fournies par des capteurs. Après ces expériences, il
nous semble clair que bien qu'il soit intéressant de fournir des données
binaires à l'utilisateur, il ne faut pas que cela lui masque des valeurs qui
pourrait apporter plus d'informations.

\paragraph{}
Il a finalement été très intéressant pour nous d'expérimenter à quel point
certains aspects qui nous semble simples dans les interactions humaines
peuvent se révéler difficile à exprimer à l'aide d'un robot.
