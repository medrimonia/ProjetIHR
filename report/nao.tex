% mini intro
\subsubsection{Expression}
\paragraph{}
Dans le cadre de l'expression d'émotions, les principales faiblesses du Nao
résident dans l'absence de degrés de liberté dans le visage et
dans la faible amplitude de ceux disponibles. Effectivement, bien qu'il soit
capable de lui faire émettre des sons, il n'est pas possible de jouer avec 
les expressions du visage. Le Nao se situe donc aux antipodes du Kismet
\footnote{\url{http://www.ai.mit.edu/projects/sociable/baby-bits.html}} qui
ne possède pas de corps, mais permet d'exprimer de manière convaincante des
émotions. Par ailleurs, le manque de degrés de liberté dans le buste empêche
aussi certains mouvements asymétriques, permettant par exemple d'exprimer la
peur.

% Limitations des degrés de libertés
\paragraph{}
Puisqu'il n'a pas de degré de liberté dans le visage, toutes les expressions
doivent être exprimées aux travers des autres degrés de liberté. Par
conséquent il est nécessaire de pouvoir s'appuyer entièrement sur les degrés
de liberté restant. Ceux-ci présentent des amplitudes qui ne ressemblent pas
à celles d'un humain. Par conséquent, il est parfois impossible d'atteindre
des positions qui sont naturels chez un humain. Par exemple, le manque
d'amplitude dans le coude empêche le robot de venir se frotter les yeux avec
les doigts pour simuler la tristesse. De plus, lorsque l'on souhaite
exprimer des sentiments présentant une dynamique assez forte, comme
l'excitation, les moteurs se révèlent trop lent ou leur contrôle trop lissé
pour permettre des mouvements demandant une forte accélération.

% Stabilité
\paragraph{}
L'obligation d'utiliser le corps pour exprimer les émotions engendre aussi
des problèmes dont on aimerait pouvoir se passer lorsque l'on se concentre
sur l'expression de celles-ci. Effectivement, il faut toujours veiller à
conserver l'équilibre du robot qui devient assez précaire dès qu'il bouge.

% Prosodie
\paragraph{}
Ces problèmes d'expression des émotions seraient moins gênant si le
générateur de voix permettait d'intégrer des aspects prosodique.
Malheureusement, cet aspect étant encore en pleine recherche, il est naturel
qu'il ne soit pas implémenté. En revanche, il semble difficile de prétendre
à une vraie expressivité sur un robot qui ne dispose pas de degrés de
liberté dans le visage tant qu'une avancée notable n'aura pas été faîte
dans ce domaine.

\subsubsection{Perception}
% Reconnaissance vocale boiteuse
\paragraph{}
Bien que le Nao possède un système de reconnaissance vocale, la qualité de
celui-ci reste discutable et il est donc relativement difficile de lui
faire comprendre quelque chose. La modalité sonore qui nous est
naturelle est donc difficile à exploiter concrètement.

% Capteurs binaires
\paragraph{}
Bien que le Nao soit équipé de capteurs de pression permettant de simuler le
sens du toucher, le fait que ceux-ci renvoient une réponse binaire est
relativement gênant. Il n'est pas possible de percevoir la nature du
contact~: Est-ce que quelqu'un prend le robot par la main ou est-ce qu'il
vient de le frapper? Est-ce que la prise est ferme ou relâchée? En l'absence
de réponse à ces questions, les capteurs ne sont plus si différents de
boutons programmables, comme on pourrait en trouver sur une télécommande.
Le sens associé à chaque capteur doit donc être programmé et il en ressort
une interaction qui ne semble pas naturelle.

% Capteurs visuels
\paragraph{}
La modalité la plus performante pour percevoir le monde extérieur avec le Nao
reste l'utilisation de la caméra. Effectivement, celle-ci permet de reconnaître
des objets avec une assez grande fiabilité. Cependant, comme il est nécessaire
d'apprendre au Nao à reconnaître l'objet auparavant, la manière d'interagir
restent relativement codifiée. Il est donc difficile d'agir avec un Nao sans le
voir avant tout comme une machine. L'aspect négatif de ce fait est que
l'interaction étant codifiée à l'extrême, il reste peu de place pour une
interaction libre.
