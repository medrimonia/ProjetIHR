\paragraph{}
Le but de ce projet était de simuler des émotions pertinentes au cours d'une
interaction entre un Nao et un humain. Pour ce faire, il était nécessaire
de développer différents modules à bord du Nao.

\paragraph{}
Afin que le Nao agisse en adéquation avec son environnement, il était
primordial qu'il ait une certaine perception du monde extérieur. Que ce soit
par des capteurs de pression, une caméra ou des micros, il est essentiel que
celui-ci puisse recueillir des informations.

\paragraph{}
Ces perceptions engendreront des comportement par le biais d'un modèle
comportemental qui ne se contentera pas de réagir, mais prédira aussi les
suites possibles d'une perception grâce à un module d'apprentissage.

\paragraph{}
Enfin, ces comportements souhaités seront appliqués par le biais de
mouvements du robot et de sons qu'ils produit afin que l'humain qui
interagit avec le Nao ait l'impression que le robot est en train de ressentir
une émotion.

\paragraph{}
Au terme de ce rapport, nous pourrons conclure sur les difficultés que nous
avons rencontrées lors de la création de ces différents module. Nous
discuterons aussi des compétences et du savoir que nous a apporté ce projet.
