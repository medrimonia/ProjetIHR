\subsection{Question 1 : Test de Turing}
% Description du test
\paragraph{}
Le test de Turing a été proposé par Alan Turing en 1950. Le but était de
pouvoir rapidement évaluer si un programme était intelligent sans avoir à
établir une longue liste de critères à remplir pour accéder à ce statut. Le
principe de ce test est simple~: un humain va poser des questions par écrit en
cherchant à déterminer grâce aux réponses si son interlocuteur est un ordinateur
ou un humain. S'il n'y parvient pas, le test de Turing est réussi.

% Distinction comportement humain intelligent / stupide
\paragraph{}
Un point primordial à évoquer est que ce test ne se base pas sur la performance d'un
programme pour effectuer une tâche particulière, il concerne plutôt la capacité
d'un ordinateur à se faire passer pour un humain. Cette différence peut être
comprise très facilement par le fait que si on demande de trouver le plus
court chemin dans un graphe à 2000 sommets et que la réponse est juste et
pratiquement instantanée, il est évident que l'interlocuteur n'est pas un
humain.

% Introduction d'erreur pour paraître réaliste
\paragraph{}
Afin de mieux couvrir ces erreurs que les humains sont capables de faire,
certains programmes ajoutent volontairement des fautes de frappes à leurs
réponses aléatoirement, ce qui a contribué à les rendre plus crédibles lors de
concours de tests de Turing.

% Bots jeux vidéos
\paragraph{}
La problèmatique d'obtenir des programmes capables d'agir comme des humains est
aussi essentiel dans les jeux vidéos. Une catégorie particulièrement férue de
bots est le FPS\footnote{First-person shooter : Jeu de tir à la première
personne}. Effectivement, il est facile d'imaginer dans ce type de jeu des bots
qui tirent toujours aux bons endroits, quelque soit la distance et ne manquent
que très rarement leur cible si le projectile est assez rapide. Cependant, ce
genre de comportement n'est pas du tout humain et très peu divertissant. Si même
les meilleurs joueurs du monde se font écraser par les bots, leur intérêt lors
de parties avec des joueurs humains est limité. Il existe même un concours
\footnote{\url{http://botprize.org/}} qui consistait à placer des bots et des
humains dans la même partie et à demander aux humains de distinguer les joueurs
des intelligences artificielles.

% capacité nécessaire / test du turing complet
\paragraph{}
Si le test de turing dans sa forme originale fait appel à du traitement du
langage, à de l'analyse sémantique ainsi qu'à de l'apprentissage. Il existe une
autre version plus adaptée aux robots~: le test de turing complet. Celui-ci
inclut un signal vidéo afin de tester les perceptions de l'interlocuteur, ainsi
qu'une capacité à manipuler des objets que l'examinateur pourrait lui fournir.

% Exemple 1 : Question balèze!
\paragraph{}
\begin{centering}
  {\em ``Quelle est la valeur de la racine de 25335, une réponse à la louche est
    demandée?''}
\end{centering}

\paragraph{}
Voilà la première question que nous pourrions poser afin de déterminer si nous
avons à faire à un humain ou un ordinateur. Effectivement, afin de donner une
réponse crédible à cette question, il est nécessaire qu'un programme réussisse
à répondre avec un degré de précision correspondant à l'expression utilisée et
aussi après un temps raisonnable pour un humain et non pour un ordinateur. Cette
question permettrait donc de détecter tout ordinateur qui serait trop performant
par rapport à un humain.

% Exemple 2 : Mélange de lettres + sentiment
\paragraph{}
\begin{centering}
  {\em ``Quelle est est l'hitsiore qui t'ait rnedu le plus trsite?''}
\end{centering}

\paragraph{}
Cette seconde question fait appel à une capacité humaine assez difficile à
exprimer pour les ordinateurs puisqu'elle fait appel au contenu émotionnel du
passé de l'interlocuteur. Simultanément elle fait appel à une capacité un peu
plus poussée du traitement du langage, puisque l'interlocuteur ne doit pas
seulement être capable de traiter un langage classique, mais un langage avec des
lettres inversées. Ce traitement qui est effectué assez facilement et très
naturellement par les humains est en revanche plus compliqué sur un plan
de traitement automatisé par un programme.

\subsection{Question 2 : Nao}
% mini intro
\subsubsection{Expression}
\paragraph{}
Dans le cadre de l'expression d'émotions, les principales faiblesses du Nao
résident dans l'absence de degrés de liberté dans le visage et
dans la faible amplitude de ceux disponibles. Effectivement, bien qu'il soit
capable de lui faire émettre des sons, il n'est pas possible de jouer avec 
les expressions du visage. Le Nao se situe donc aux antipodes du Kismet
\footnote{\url{http://www.ai.mit.edu/projects/sociable/baby-bits.html}} qui
ne possède pas de corps, mais permet d'exprimer de manière convaincante des
émotions. Par ailleurs, le manque de degrés de liberté dans le buste empêche
aussi certains mouvements asymétriques, permettant par exemple d'exprimer la
peur.

% Limitations des degrés de libertés
\paragraph{}
Puisqu'il n'a pas de degré de liberté dans le visage, toutes les expressions
doivent être exprimées aux travers des autres degrés de liberté. Par
conséquent il est nécessaire de pouvoir s'appuyer entièrement sur les degrés
de liberté restant. Ceux-ci présentent des amplitudes qui ne ressemblent pas
à celles d'un humain. Par conséquent, il est parfois impossible d'atteindre
des positions qui sont naturels chez un humain. Par exemple, le manque
d'amplitude dans le coude empêche le robot de venir se frotter les yeux avec
les doigts pour simuler la tristesse. De plus, lorsque l'on souhaite
exprimer des sentiments présentant une dynamique assez forte, comme
l'excitation, les moteurs se révèlent trop lent ou leur contrôle trop lissé
pour permettre des mouvements demandant une forte accélération.

% Stabilité
\paragraph{}
L'obligation d'utiliser le corps pour exprimer les émotions engendrent aussi
des problèmes dont on aimerait pouvoir se passer lorsque l'on se concentre
sur l'expression de celles-ci. Effectivement, il faut toujours veiller à
conserver l'équilibre du robot qui devient assez précaire dès qu'il bouge.

% Prosodie
\paragraph{}
Ces problèmes d'expression des émotions seraient moins gênant si le
générateur de voix permettait d'intégrer des aspects prosodique.
Malheureusement, cet aspect étant encore en pleine recherche, il est naturel
qu'il ne soit pas implémenté. En revanche, il semble difficile de prétendre
à une vraie expressivité sur un robot qui ne dispose pas de degrés de
liberté dans le visage tant qu'une avancée notable n'aura pas été faîte
dans ce domaine.

\subsubsection{Perception}
% Reconnaissance vocale boiteuse
\paragraph{}
Bien que le Nao possède un système de reconnaissance vocale, la qualité de
celui-ci reste discutable et il est donc relativement difficile de lui
faire comprendre quelque chose. La modalité sonore qui nous est
naturelle est donc difficile à exploiter concrètement.

% Capteurs binaires
\paragraph{}
Bien que le Nao soit équipé de capteurs de pression permettant de simuler le
sens du toucher, le fait que ceux-ci renvoient une réponse binaire est
relativement gênant. Il n'est pas possible de percevoir la nature du
contact~: Est-ce que quelqu'un prend le robot par la main ou est-ce qu'il
vient de le frapper? Est-ce que la prise est ferme ou relâchée? En l'absence
de réponse à ces questions, les capteurs ne sont plus si différents de
boutons programmables, comme on pourrait en trouver sur une télécommande.
Le sens associé à chaque capteur doit donc être programmé et il en ressort
une interaction qui ne semble pas naturelle.

% Capteurs visuels
\paragraph{}
La modalité la plus performante pour percevoir le monde extérieur avec le Nao
reste l'utilisation de la caméra. Effectivement, celle-ci permet de reconnaître
des objets avec une assez grande fiabilité. Cependant, comme il est nécessaire
d'apprendre au Nao à reconnaître l'objet auparavant, la manière d'interagir
restent relativement codifiée. Il est donc difficile d'agir avec un Nao sans le
voir avant tout comme une machine. L'aspect négatif de ce fait est que
l'interaction étant codifiée à l'extrême, il reste peu de place pour une
interaction libre.


\subsection{Question 3 : Cobotique}

%\textbf{Robot capable d'interagir avec l'humain, dans un environnement humain, sans
%risque pour l'humain.}

\paragraph{}
La cobotique fait référence à la collaboration entre un humain et un robot.
Ce terme vient de la contraction de l'anglicisme \emph{collaborative-robot}.

\paragraph{}
La cobotique possède de nombreux avantages par rapport à l'intervention de
l'homme seul, en le soulageant dans l'exécution de tâches pénibles - voire
impossible à réaliser par un humain. Les exosquelettes correspondent
parfaitement à ce cas d'utilisation.
De l'autre côté, l'association de l'homme et de la machine va donner de la
valeur ajoutée par rapport à l'automatisation complète du fait des capacités
visuelles et manuelles de l'homme qui sont indéniables et du fait de la grande
adaptabilité de l'homme.

\paragraph{}
Cependant une question se pose avec cette collaboration de l'homme avec la
machine : quelle portion d'une tâche doit être réalisée par le robot et quelle
portion doit être laissée à l'homme ?

\paragraph{}
Cette question du positionnement du curseur entre le tout automatique et le
contrôle complet laissé à l'humain est déterminante lors de la conception d'un
robot et doit être convenablement réfléchie afin d'obtenir au résultat optimal
en fonction du contexte et des besoins.
