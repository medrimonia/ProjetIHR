\subsection{Question 1 : Test de Turing}
% Description du test
% Distinction comportement humain intelligent / stupide
% DeepBlue -> new kind of intelligence
% Introduction d'erreur pour paraître réaliste
% Bots jeux vidéos (quake)

% Exemples : ...

\subsection{Question 2 : Nao}

\subsection{Question 3 : Cobotique}

\textbf{Robot capable d'interagir avec l'humain, dans un environnement humain, sans
risque pour l'humain.}

La cobotique fait référence à la collaboration entre un humain et un robot.
Ce terme vient de la contraction de l'anglicisme \emph{collaborative-robot}.

La cobotique possède de nombreux avantages par rapport à l'intervention de
l'homme seul, en le soulageant dans l'exécution de tâches pénibles - voire
impossible à réaliser par un humain. Les exosquelettes correspondent
parfaitement à ce cas d'utilisation.
De l'autre côté, l'association de l'homme et de la machine va donner de la
valeur ajoutée par rapport à l'automatisation complète du fait des capacités
visuelles et manuelles de l'homme qui sont indéniables et du fait de la grande
adaptabilité de l'homme.
