\subsection{Question 1 : Test de Turing}
% Description du test
\paragraph{}
Le test de Turing a été proposé par Alan Turing en 1950. Le but était de
pouvoir rapidement évaluer si un programme était intelligent sans avoir à
établir une longue liste de critères à remplir pour accéder à ce statut. Le
principe de ce test est simple~: un humain va poser des questions par écrit en
cherchant à déterminer grâce aux réponses si son interlocuteur est un ordinateur
ou un humain. S'il n'y parvient pas, le test de Turing est réussi.

% Distinction comportement humain intelligent / stupide
\paragraph{}
Un point primordial à évoquer est que ce test ne se base sur la performance d'un
programme pour effectuer une tâche particulière, il concerne plutôt la capacité
d'un ordinateur à se faire passer pour un humain. Cette différence peut être
comprise très facilement par le fait que si on demande de trouver le plus
court chemin dans un graphe à 2000 sommets et que la réponse est juste et
pratiquement instantanée, il est évident que l'interlocuteur n'est pas un
humain.

% Introduction d'erreur pour paraître réaliste
\paragraph{}
Afin de mieux couvrir ces erreurs que les humains sont capables de faire,
certains programmes ajoutent volontairement des fautes de frappes à leurs
réponses aléatoirement, ce qui a contribué à les rendre plus crédibles lors de
concours de tests de Turing.

% Bots jeux vidéos
\paragraph{}
La problèmatique d'obtenir des programmes capables d'agir comme des humains est
aussi essentiel dans les jeux vidéos. Une catégorie particulièrement férue de
bots est le FPS\footnote{First-person shooter : Jeu de tir à la première
personne}. Effectivement, il est facile d'imaginer dans ce type de jeu des bots
qui tirent toujours aux bons endroits, quelque soit la distance et ne manquent
que très rarement leur cible si le projectile est assez rapide. Cependant, ce
genre de comportement n'est pas du tout humain et très peu divertissant. Si même
les meilleurs joueurs du monde se font écraser par les bots, leur intérêt lors
de parties avec des joueurs humains est limité. Il existe même un concours
\footnote{\url{http://botprize.org/}} qui consistait à placer des bots et des
humains dans la même partie et à demander aux humains de distinguer les joueurs
des intelligences artificielles.

% capacité nécessaire / test du turing complet
\paragraph{}
Si le test de turing dans sa forme originale fait appel à du traitement du
langage, à de l'analyse sémantique ainsi qu'à de l'apprentissage. Il existe une
autre version plus adaptée aux robots~: le test de turing complet. Celui-ci
inclut un signal vidéo afin de tester les perceptions de l'interlocuteur, ainsi
qu'une capacité à manipuler des objets que l'examinateur pourrait lui fournir.

% Exemple 1 : Question balèze!
\begin{centering}
  {\em Quelle est la valeur de la racine de 25335, une réponse à la louche est
    demandée?}
\end{centering}

\paragraph{}
Voilà la première question que nous pourrions poser afin de déterminer si nous
avons à faire à un humain ou un ordinateur. Effectivement, afin de donner une
réponse crédible à cette question, il est nécessaire qu'un programme réussisse
à répondre avec un degré de précision correspondant à l'expression utilisée et
aussi après un temps raisonnable pour un humain et non pour un ordinateur. Cette
question permettrait donc de détecter tout ordinateur qui serait trop performant
par rapport à un humain.

% Exemple 2 : Mélange de lettres + sentiment
\begin{centering}
  {\em Quelle est est l'hitsiore qui t'ait rnedu le plus trsite?}
\end{centering}

\paragraph{}
Cette seconde question fait appel à une capacité humaine assez difficile à
exprimer pour les ordinateurs puisqu'elle fait appel au contenu émotionnel du
passé de l'interlocuteur. Simultanément elle fait appel à une capacité un peu
plus poussée du traitement du langage, puisque l'interlocuteur ne doit pas
seulement être capable de traiter un langage classique, mais un langage avec des
lettres inversées. Ce traitement qui est effectué assez facilement et très
naturellement par les humains est en revanche plus compliqué sur un plan
de traitement automatisé par un programme.

\subsection{Question 2 : Nao}

\subsection{Question 3 : Cobotique}
