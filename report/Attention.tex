% Récup capteur

\paragraph{}
Grâce à l'interface de programmation choregraph et de son API , il nous a été
facile d'accéder aux différents capteurs, cependant le traitement qui était
effectué ne nous convenait pas. En effet, les boîtes fournies par le programme
renvoyaient un signal de type bang pour signaler le passage à l'état actif du
capteur. Ce type de signal ne convenait pas dans notre architecture logicielle
qui demandait que l'on sache en permanence les différents états de chaque
groupe de capteurs dans le but de pouvoir déterminer un stimulus dominant.

Nous voulions aussi éviter les rebonds au niveau des capteurs tactiles. 
De même, pour la reconnaissance d'objet nous avons constaté des ``faux-négatif''
correspondant par exemple aux mouvements. 
  
C'est pourquoi nous avons refait nos propres boîtes concernant les capteurs.
Ces boîtes scrutent les capteurs de manière à renvoyer un doublet de la forme :
\begin{itemize}
  \item Le nom du stimulus
  \item La valeur~: 1 si un des capteurs est activé , -1 sinon
\end{itemize}
 
De plus nous avons fait le choix d'utiliser les capteurs binaires par zone 
(Mains et tête) comme un seul capteur. Nous avions donc 2 stimuli fixes et
des stimuli variables (reconnaissance d'objet). 

\paragraph{}
Toutes les boîtes des capteurs sont alors reliées à une boîte centralisant les 
données. C'est cette boîte qui se charge d'envoyer les informations nécessaires 
à l'entité chargée de déterminer la réaction à avoir.
Grâce aux informations reçues elle garde en mémoire en permanence l'état de chaque capteur.

L'interface entre la perception et le modèle de comportement avait un cahier des charges assez précis. 
En effet le modèle de comportement étant le centre de l'intelligence, toute l'interface 
entre les différents blocs a été décidée par rapport à ses besoins. Nous avions donc définit
une interface simple se basant sur le passage de chaine de caractère. Le but était 
d'envoyer une chaine de caractère correspondant au stimulus dominant. Les stimuli ne 
sont pas envoyés périodiquement mais seulement quand il y a un changement dans le 
stimulus dominant.

\paragraph{}
Afin de déterminer le stimulus dominant nous avons mis en place un système de score. Un timer a 
été connecté à notre boîte centralisant les informations. \'A chaque tic du timer, la boîte regarde
l'état courant des stimuli (si ils sont actuellement actifs ou non) et augmente les scores des stimuli
actifs et diminue celui des non-actifs.

L'envoi du simulus ayant le plus grand score pouvait poser certains problèmes. Il nous a donc 
fallu modifier le choix du stimulus dominant pour ne pas envoyer en trois coups d'horloge
présence, absence, présence ce qui aurait pu fausser le modèle prédictif. 
Nous avons donc décidé de créer une hystérésis sur les différentes mesures de
capteurs pour filtrer les faux-positifs et les faux-négatifs éventuels. 

\paragraph{}
Dans les paramètres de la boîte, il est possible de fixer deux seuils.
Le premier seuil permet de définir à partir de quel score un stimulus devient dominant.
Si un stimulus franchi ce seuil alors qu'un autre stimulus est déjà dominant rien ne se passent.
Tant que le stimulus dominant n'est pas repassé en dessous du deuxième seuil il conserve ça place de 
stimulus dominant.
Ce mécanisme permet alors d'éviter les oscillations sur le stimulus dominant.

