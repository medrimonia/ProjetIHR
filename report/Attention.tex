% Récup capteur

\paragraph{}
Grâce à l'interface de programmation choregraph et de son API , il nous a été
facile d'accéder aux différents capteurs, cependant le traitement qui était
effectué ne nous convenait pas. En effet, les boites fournies pas le programme
renvoyaient un signal de type bang pour signaler le passage à l'état actif du
capteur. Ce type de signal ne convenait pas dans notre architecture logicielle
qui demandait que l'on sache en permanence les différents états de chaque
groupe de capteurs. De plus nous avons fait le choix d'utiliser les capteurs
binaire par zone (Mains et tête) comme un seul capteur. Nous avions donc 2 stimuli fixes et
des stimuli variables (reconnaissance d'objet). Nous avons donc modifié les
différents blocs pour qu'ils nous retournent un tuple comprenant :
\begin{itemize}
  \item Le nom du groupe de capteur
  \item La valeur~: 1 si un des capteurs est activé , -1 sinon
\end{itemize}

\paragraph{}
Ces informations étaient transférées par chaque bloc de zone à une boite qui
s'occupait de centraliser l'envoi des données à l'entité chargée de déterminer la
réaction à avoir.

\paragraph{}
Cette boite avait un cahier des charges assez précis. En effet la décision de
comportement étant le centre de l'intelligence, toute la réalisation des signaux
entre les différents blocs a été décidée par ses besoins.

\paragraph{}
Cela a impliqué une fréquence d'envoi des données fixes pour permettre la
détermination des actions sans surcharge pour l'intelligence. De plus nous
nous sommes rendu compte que lors de l'envoi de certains capteurs(
reconnaissance d'objet notamment) nous avions des ``faux-négatif'' qui
correspondait à un mouvement de l'objet. Il nous a donc fallu modifier la
récupération des données pour ne pas envoyer en trois coup d'horloge
présence, absence, présence ce qui aurait pu fausser le modèle prédictif.
Nous avons donc décidé de créer un hystérésis sur les différentes mesures de
capteurs pour filtrer les faux-positifs et les faux-négatifs éventuels. Ce
dernier est fonction des paramètres de la boite de choregraph.

% Graphe pour l'hysteresis

\paragraph{}
L'incrémentation de ces seuils se fait sur les ticks de timer qui nous
permettent d'assurer un temps d'échantillonage égal sur chaque capteur.
Une priorité d'envoi vers l'intelligence est faite sur le premier activé puis
sur le suivant dans la zone de stockage. 
