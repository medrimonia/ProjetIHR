\subsubsection{Modèle final utilisé}
\paragraph{}
Afin de rendre plus dynamique notre modèle d'apprentissage et de faciliter la
mise à jour des connaissances, nous avons décidé d'utiliser une version
enrichie du modèle Laplacien. Pour ce faire, nous avons décidé de prendre en
compte la temporalité. 

%Caractéristiques attendues
\paragraph{}
Les caractéristiques que nous souhaitions remplir avec notre modèle prédictif
étaient les suivantes :
\begin{itemize}
\item Favoriser les éléments récents, particulièrement à court-terme
\item Prendre en compte le nombre d'occurence des expériences,
      particulièrement à moyen terme
\item Oublier les expériences très vieilles, en convergeant à long terme vers
      une loi uniforme, cette loi uniforme doit aussi être présente dans le
      cas d'une absence totale d'expériences.
\end{itemize}

% Experience
\paragraph{}
Nous avons donc décidé de définir une expérience sous
la forme suivante~:
$$x = \{s, v, t\}$$
\begin{itemize}
\item $s(x)$ le stimuli précédent $x$
\item $v(x)$ la valeur associée à $x$
\item $t(x)$ le temps auquel s'est produit $x$
\end{itemize}

% Ensembles
\paragraph{}
Dans le cadre de nos expériences, il existe plusieurs ensembles qui vont
servir de base à l'évaluation des expériences~:
\begin{itemize}
\item $S=\{caresse, claque, mouchoir, ...\}$~: L'ensemble des stimulis.
\item $V=\{+,-,rien\}$~: Les valeurs associées aux stimulis.
\item $E$~: L'ensemble des expériences
\item $E(s)$ avec $s \in S$~: L'ensemble des expériences précédées par le
      stimuli $s$.
\item $E(s,v)$ avec $s \in S$ et $v \in V$~: L'ensemble des expériences
      précédées par le stimuli $s$ avec comme valeur associée $v$.
\end{itemize}

% Probabilités
\paragraph{}
On notera:
\begin{itemize}
\item $P(s,t)$~: La probabilité du stimuli $s$ à l'instant $t$.
\item $P(v|s, t)$~: La probabilité que la valeur $v$ apparaisse en fonction
      de $t$, sachant que le dernier stimuli perçu était $s$.
\end{itemize}
Afin de débuter par une loi uniforme, nous avons gardé une loi ayant une
forme commune avec celle du lissage de Laplace.
