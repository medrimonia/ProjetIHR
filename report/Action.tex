\paragraph{}
Notre modèle de décision étant maintenant opérationnel, il nous faut définir une
réaction appropriée dans chaque cas. Pour cela nous allons utiliser deux boites
spécifiques fournit par Chorégraphe:\\
\begin{enumerate}
  \item La boite TimeLine
  \item La boite PlaySound
\end{enumerate}

\paragraph{}
%Pourquoi ces deux boites?
Nous avons choisi d'utiliser ces deux boites pour des raisons simples, la
première nous sert à définir facilement les actions physiques. Cependant ces
dernières sont assez limitées par la géométrie de Nao, nous avons donc rajouter
une émission de son pour amplifier l'émotion.

\paragraph{}
Nous avons implémenté 5 réactions qui sont:\\
\begin{itemize}
  \item La peur
  \item Le soulagement
  \item La surprise
  \item La déception
  \item L'excitation
\end{itemize}

%Ces différentes actions sont connues par l'intelligence et nous recevons l'ordre
%de cette dernière que nous traitons ensuite en bloquant les moteurs pour
%l'action déterminé.

\paragraph{}
Nous avons décidé que l'intelligence du Nao déterminerait la réaction à
appliquer. Cependant un problème s'est posé, il nous arrivait d'envoyer une
deuxième demande de réaction avant que la première ne soit fini. Cela avait pour
effet de rendre instable certaines actions. Nous avons donc décidé de bloquer
l'appel aux moteurs tant que la réaction à une stimulation ne serait pas finie.
