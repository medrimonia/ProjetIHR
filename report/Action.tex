\paragraph{}
Notre modèle de décision étant maintenant opérationnel, il nous faut définir une
réaction appropriée dans chaque cas. Pour cela nous allons utiliser deux boites
spécifiques fournit par Chorégraphe:\\
\begin{enumerate}
  \item La boite TimeLine
  \item La boite PlaySound
\end{enumerate}

\paragraph{}
%Pourquoi ces deux boites?
Nous avons choisi d'utiliser ces deux boites pour des raisons simples, la
première nous sert à définir facilement les actions physiques. Cependant ces
dernières sont assez limitées par la géométrie de Nao, nous allons donc rajouter
une émission de son pour amplifier l'émotion.

\paragraph{}
Nous avons implémenté 5 réactions qui sont:\\
\begin{itemize}
  \item La peur
  \item Le soulagement
  \item La surprise
  \item La déception
  \item L'excitation
\end{itemize}

%Ces différentes actions sont connues par l'intelligence et nous recevons l'ordre
%de cette dernière que nous traitons ensuite en bloquant les moteurs pour
%l'action déterminé.

\paragraph{}
Nous avons décidé que l'intelligence du Nao déterminerait la réaction à
appliquer. Cependant un problème s'est posé, il arrivait que nous envoyons 
